\documentclass{article}

\usepackage{a4wide}
\usepackage{amsmath}
\usepackage{amsfonts}
\usepackage[latin1]{inputenc} %input font encoding
\usepackage[T1]{fontenc} % output font encoding
\usepackage{listings}
\usepackage{hyperref}

\hypersetup{
    colorlinks,
    citecolor=black,
    filecolor=black,
    linkcolor=black,
    urlcolor=black
}

\title{Visual Epidemic Simulation}
\author{Matthew Lakin}
\date{October 2024 - May 2025}

\begin{document}
\maketitle

\begin{center}
    \vspace*{\fill}
    \textbf{\Large{Project Supervisor: Dr. Hossein Nevisi}}
    \vspace*{\fill}
\end{center}
\thispagestyle{empty}

\newpage

\tableofcontents


\newpage

\section{Abstract}
This project intends to create a visual simulation of an epidemic. Using publically available data from the COVID-19 epidemic, the simulation will represent the data in a visual format similar to a dashboard. The simulation will be created using a Dockerised Java environment, with the simulation itself being written in Java.
\newpage

\section{Introduction}
\newpage

\section{Problem Domain}
\subsection{Minimum Viable Product}
I have identified the minimal requirements for this project to be successful.
\begin{enumerate}
    \item Front end
    \begin{enumerate}
        \item The front end must show a map of the Earth.
        \item The map will render polygon and country data.
        \item The front end must show a timeline of the epidemic.
        \item The individual countries must be clickable.
        \item Clicking a country will show the number of cases and deaths per week.
        \item The user must be able to scrub through the timeline.
        \item The user must be able to select data from a CSV file to be displayed.
    \end{enumerate}
    \item Back end
    \begin{enumerate}
        \item The back end will be using a node.js server.
        \item The epidemic data will be stored in a csv file.
        \item The polygon data will be stored in a GeoJSON file.
        \item The node.js server will be dockerised.
    \end{enumerate}
\end{enumerate}
\newpage
\subsection{Stretch Goals}
For this project, the following requirements have been identified:
\begin{enumerate}
    \item The application must be able to input premade epidemic data.
    \begin{enumerate}
        \item The data will be in the form of a CSV file.
        \item The data will contain the country, date, number of cases, number of deaths per week.
        \item The data will be publically available from the COVID-19 epidemic.
    \end{enumerate}
    \item The application must be able to display the data in a visual format.
    \begin{enumerate}
        \item The data will be displayed in a dashboard format.
        \item The map of the Earth will use mapping data from GeoJSON.
        \item The user will be able to scrub through the data to see the progression of the epidemic.
        \item The user will be able to click each country for a more detailed view.
        \item The user will be able to see the number of cases and deaths per week.
    \end{enumerate}
\end{enumerate}
\newpage

\section{Methodology}
\textbf{\LARGE{Waterfall}}
\newpage

\section{Technical Solution}
\newpage

\section{Results and Analysis}
\newpage

\section{Conclusion}
\newpage




\end{document}